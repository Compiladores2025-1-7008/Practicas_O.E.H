\documentclass{article}
\usepackage{graphicx} % Required for inserting images
\usepackage[spanish,mexico]{babel}

\usepackage{tikz}
\usepackage{algorithm}
\usepackage{algorithmic}

\usepackage[letter, left=2.0cm, right=2.0cm, top=2.0cm, bottom=2.0cm]{geometry}
\usepackage{amsmath}
\usepackage{blindtext}
\usepackage{hyperref}
\usepackage[utf8]{inputenc}
\usetikzlibrary{automata, positioning}

\title{ Universidad Nacional Autónoma de México \\ Facultad de Ciencias \\ Compiladores \\ Práctica 3}

\author{ Osorio Escandón Huriel | 317204652 | Huriel117@ciencias.unam.mx }

\date{}

\begin{document}
\maketitle

\begin{enumerate}
    \item Determinar los conjuntos $ N, \Sigma $ y el símbolo inicial $ S $.

    \textbf{Conjuntos de la Gramática G}

    \textbf{Conjunto de No Terminales $(N)$}

    El conjunto de no terminales es:

    \[
    N = \{ \text{programa, declaraciones, declaracion, tipo, lista\_var, sentencias, sentencia, expresion} \}
    \]

    \textbf{Conjunto de Terminales $(\Sigma)$}

    El conjunto de terminales es:

    \[
    \Sigma = \{ \text{int, float, ,, =, ;, (, ), +, -, *, /, if, else, while, identificador, numero} \}
    \]

    \textbf{Símbolo Inicial $(S)$}

    El símbolo inicial de la gramática es:

    \[
    S = \text{programa}
    \]

    \item Mostrar en el archivo el proceso de eliminación de ambigüedad o justificar, en caso de no ser necesario.

    \textbf{Eliminación de Ambigüedad}

    \textbf{Expresiones Aritméticas}

    La gramática original para las expresiones aritméticas presenta una potencial ambigüedad, ya que no especifica la precedencia ni la asociatividad de los operadores \texttt{+}, \texttt{-}, \texttt{*} y \texttt{/}. Para resolver esta ambigüedad, se van a introducir reglas que establecen una precedencia adecuada y una asociatividad por la izquierda.

    La producción original:

        \begin{verbatim}
    expresion → expresion + expresion
              | expresion - expresion
              | expresion * expresion
              | expresion / expresion
              | identificador
              | numero
              | ( expresion )
    \end{verbatim}

    Las modificaciones son las siguientes:

    \begin{verbatim}
    expresion → expresion_suma
    expresion_suma → expresion_suma + expresion_mult | expresion_suma - expresion_mult | 
    expresion_mult
    expresion_mult → expresion_mult * expresion_unaria | expresion_mult / expresion_unaria | 
    expresion_unaria
    expresion_unaria → identificador | numero | ( expresion )
    \end{verbatim}

    De esta manera, garantizamos que \texttt{*} y \texttt{/} tienen mayor precedencia que \texttt{+} y \texttt{-}, y todas las operaciones son asociativas por la izquierda.

    \textbf{Problema del if-else colgante}

    La gramática original presenta el problema del \textit{if-else colgante}, que puede llevar a ambigüedades en la interpretación de las sentencias. Para resolverlo, se harán las siguientes modificaciones:

    \begin{verbatim}
    sentencia → sentencia_if | sentencia_otra
    sentencia_if → if ( expresion ) sentencia_if else sentencia_if 
                 | if ( expresion ) sentencia_otra
    sentencia_otra → identificador = expresion ; | while ( expresion ) sentencias | { sentencias }
    \end{verbatim}

    De esta manera, garantizamos que un \texttt{else} siempre se empareja con el \texttt{if} más cercano que lo requiere, eliminando así la ambigüedad.

    \item Mostrar en el archivo el proceso de eliminación de la recursividad izquierda o justificar, en caso de no ser necesario. 

    \textbf{Eliminación de la Recursividad Izquierda}

    En la gramática original, se presentan varias producciones con recursividad izquierda, lo que puede causar problemas en un analizador sintáctico de descenso recursivo. 

    \textbf{Producción de \texttt{declaraciones}}

    La producción original es recursiva a la izquierda:

    \begin{verbatim}
    declaraciones → declaraciones declaracion | declaracion
    \end{verbatim}

    La nueva producción, sin recursividad izquierda, es:

    \begin{verbatim}
    declaraciones → declaracion declaraciones'
    declaraciones' → declaracion declaraciones' | ε
    \end{verbatim}

    \textbf{Producción de \texttt{lista\_var}}

    La producción original:

    \begin{verbatim}
    lista_var → lista_var , identificador | identificador
    \end{verbatim}

    La nueva producción sin recursividad izquierda:

    \begin{verbatim}
    lista_var → identificador lista_var'
    lista_var' → , identificador lista_var' | ε
    \end{verbatim}

    \textbf{Producción de \texttt{sentencias}}

    La producción original:

    \begin{verbatim}
    sentencias → sentencias sentencia | sentencia
    \end{verbatim}

    La nueva producción sin recursividad izquierda:

    \begin{verbatim}
    sentencias → sentencia sentencias'
    sentencias' → sentencia sentencias' | ε
    \end{verbatim}

    \item Mostrar en el archivo el proceso de factorización izquierda o justificar, en caso de no ser necesario.

    \textbf{Factorización Izquierda}

    Después de analizar las producciones de la gramática, notemos que no es necesario realizar factorización izquierda, ya que no existen prefijos comunes entre las alternativas de las producciones.

    \textbf{Producción de \texttt{expresion}}

    La producción original para \texttt{expresion} no contiene prefijos comunes:

    \begin{verbatim}
    expresion → expresion + expresion
              | expresion - expresion
              | expresion * expresion
              | expresion / expresion
              | identificador
              | numero
              | ( expresion )
    \end{verbatim}

    \textbf{Producción de \texttt{sentencia}}

    La producción de \texttt{sentencia} tampoco presenta prefijos comunes:

    \begin{verbatim}
    sentencia → identificador = expresion ;
            | if ( expresion ) sentencias else sentencias
            | while ( expresion ) sentencias
    \end{verbatim}

    \textbf{Producción de \texttt{declaraciones}}

    Las producciones para \texttt{declaraciones} tampoco requieren factorización izquierda, ya que no hay prefijos comunes:

    \begin{verbatim}
    declaraciones → declaracion declaraciones'
    declaraciones' → declaracion declaraciones' | ε
    \end{verbatim}

    Tomando en cuenta lo anterior, se tiene que, no es necesario realizar la factorización izquierda en la gramática.

    \item Mostrar en el archivo los nuevos conjuntos $N$ y $P$.

    \textbf{Nuevos Conjuntos $N$ y $P$}

    Después de realizar las modificaciones necesarias, los conjuntos de no terminales y producciones son los siguientes:

    \textbf{Conjunto de No Terminales $(N)$}

    El conjunto de no terminales es:

    $ N = \{ $programa, declaraciones, declaraciones', declaracion, tipo, lista\_var, lista\_var', sentencias, sentencias', sentencia, expresion, expresion\_suma, expresion\_mult, expresion\_unaria $ \} $

    \textbf{Conjunto de Producciones $(P)$}

    Las producciones de la gramática son las siguientes:

    \begin{verbatim}
    programa → declaraciones sentencias

    declaraciones → declaracion declaraciones'
    declaraciones' → declaracion declaraciones' | ε

    declaracion → tipo lista_var ;

    tipo → int | float

    lista_var → identificador lista_var'
    lista_var' → , identificador lista_var' | ε

    sentencias → sentencia sentencias'
    sentencias' → sentencia sentencias' | ε

    sentencia → identificador = expresion ;
              | if ( expresion ) sentencias else sentencias
              | while ( expresion ) sentencias

    expresion → expresion_suma
    expresion_suma → expresion_suma + expresion_mult | expresion_suma - expresion_mult | 
    expresion_mult 
    expresion_mult → expresion_mult * expresion_unaria | expresion_mult / expresion_unaria | 
    expresion_unaria
    expresion_unaria → identificador | numero | ( expresion )
    \end{verbatim}
    
\end{enumerate}
\end{document}